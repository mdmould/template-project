\documentclass[
    aps,
    prl,
    reprint,
    %groupedaddress,
    superscriptaddress,
    linenumbers,
    nolongbibliography,
    nobibnotes,
    nofootinbib,
    %floatfix,
]{revtex4-2}

\bibliographystyle{apsrev4-2}

\usepackage[dvipsnames, usenames]{xcolor}
\definecolor{linkcolor}{rgb}{0.1, 0.5, 0.7}

\usepackage[
	colorlinks=true,
	linkcolor=linkcolor,
	citecolor=linkcolor,
	filecolor=linkcolor,
	urlcolor=linkcolor,
	pdfusetitle,
]{hyperref}

\usepackage{amssymb}
\usepackage{amsmath}
\usepackage{physics}
\usepackage{orcidlink}

\usepackage{titlesec}
\titleformat{\section}[runin]{\it}{\thesection}{0pt}{\phantomsection}[---]
\titlespacing{\section}{\parindent}{\parskip}{-\parskip}

\let\oldacknowledgments\acknowledgments
\renewcommand{\acknowledgments}{\vspace{\baselineskip}\oldacknowledgments}

\newcommand{\comment}[1]{}
\newcommand{\rednote}[1]{{\color{Red} #1}}
\newcommand{\mm}[1]{{\color{cyan}MM: #1}}

\newcommand{\ligo}{\affiliation{LIGO Laboratory, Massachusetts Institute of Technology, Cambridge, MA 02139, USA}}
\renewcommand{\mit}{\affiliation{Kavli Institute for Astrophysics and Space Research and Department of Physics, Massachusetts Institute of Technology, Cambridge, MA 02139, USA}}


\begin{document}


\title{Title}

\author{Matthew Mould\,\orcidlink{0000-0001-5460-2910}}
\email{mmould@mit.edu}
\ligo\mit

\date{\today}

\begin{abstract}

This is the abstract.

\end{abstract}

\maketitle


\section{Introduction}

This is the introduction.


\acknowledgments

We thank friends for discussions.
%
M.M. is supported by the LIGO Laboratory through the National Science Foundation awards PHY-1764464 and PHY-2309200.
%
The authors are grateful for computational resources provided by
the LIGO Laboratory supported by National Science Foundation Grants PHY-0757058 and PHY-0823459,
and by subMIT at MIT Physics.
%
This material is based upon work supported by NSF's LIGO Laboratory which is a major facility fully funded by the National Science Foundation and has made use of data or software obtained from the Gravitational Wave Open Science Center (gwosc.org), a service of the LIGO Scientific Collaboration, the Virgo Collaboration, and KAGRA.


\bibliography{draft}


\end{document}
